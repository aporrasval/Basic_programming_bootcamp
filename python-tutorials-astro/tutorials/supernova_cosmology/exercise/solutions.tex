\documentclass[a4paper,12pt]{article}

\usepackage{amssymb}
\usepackage{amsmath}
\usepackage{amsfonts}
\usepackage{txfonts}
\usepackage{upgreek}
\usepackage{graphicx}
\usepackage{siunitx}
\usepackage{enumerate}
\usepackage[left=2cm,right=2cm,top=2cm,bottom=2cm]{geometry}

%\newcommand{\question}[2]{\textbf{\textit{#1}}\quad{\footnotesize\textit{(#2 points)}}\\[3mm]}
\newcommand{\question}[1]{\textbf{\textit{#1}}}
\newcommand{\points}[1]{\quad{\footnotesize\textit{(#1 points)}}}
\newcommand{\point}{\quad{\footnotesize\textit{(1 point)}}}
\newcommand{\HRule}{\rule{\linewidth}{0.3mm}}
\newcommand{\dd}{\mathrm{d}}
\renewcommand{\pi}{\uppi}
\newcommand{\ii}{\mathrm{i}}
\renewcommand{\thefootnote}{\normalsize\fnsymbol{footnote}}
\DeclareMathOperator{\e}{e}

\renewcommand{\theequation}{\Roman{equation}}

\begin{document}
	\pagestyle{empty}
	
	\begin{center}
		\LARGE \textbf{Astronomy from 4 perspectives: the Dark Universe}
		\HRule
	\end{center}
	\begin{flushright}
		prepared by: Florence participants and BMS
	\end{flushright}
	\begin{center}
		{\Large \textbf{exercise: Supernova-cosmology and dark energy}}\\
		\vspace*{2mm}
		{\Large \textbf{Solutions}}
		
	\end{center}
	\vspace{5mm}
	
	\begin{enumerate}[\itshape \bfseries 1.]
		
		\item \question{light-propagation in FLRW-spacetimes}\\
		Photons travel along null geodesics, $\dd s^2=0$, in any spacetime. 
		\begin{enumerate}[(a)]
			\item
			\item
			\item
			\item
			\item
		\end{enumerate}
		
		
		\item \question{light-propagation in perturbed metrics}\\
		\begin{align}
		ds^2=\left(1+2\frac{\Phi}{c^2}\right)c^2dt^2-\left(1-2\frac{\Phi}{c^2}\right)dx^2 
		\end{align}
		With $ds^2=0$:
		\begin{align}
		\left( 1+\frac{2\Phi}{c^2}\right)c^2dt^2 &= \left(1-\frac{2\Phi}{c^2}\right) dx^2\\
		\frac{dx}{dt}&=\pm c\sqrt{\frac{1+\frac{2\Phi}{c^2}}{1-\frac{2\Phi}{c^2}}}
		\end{align}
		With $\frac{1}{1-\epsilon}\approx 1+\epsilon$ for small $\epsilon$:
		\begin{align}
		\frac{dx}{dt}\approx\pm c\left(1+\frac{2\Phi}{c^2}\right)
		\end{align} 
		For a non-zero $\Phi$ this is not equal to $c$! \\
		We assign an effective index of refraction by:
		\begin{align}
		n(\Phi)=\frac{dx/dt}{c}\approx \left(1+\frac{2\Phi}{c^2}\right)
		\end{align}
		
		\item \question{classical potentials including a cosmological constant}\\
		The field equation of classical gravity including a cosmological constant $\lambda$ is given by
		\begin{equation}
		\Delta\Phi = C(n) G\rho + \lambda
		\end{equation}
		(a) field calculation\\
		Using the n-dimensional Laplacian \[\Delta=\frac{1}{r^{n-1}}\left(\frac{\partial}{\partial r}\left(r^{n-1}\frac{\partial}{\partial r}\right)+\Delta_{S^{n-1}}\right)\]
		Assuming spherical symmetry one may neglect the angular Laplace-Beltrami operator \(\Delta_{S^{n-1}}\).
		Respecting the total mass \[M=C(n)\int_0^r\textrm{d}r'\left(r'\right)^{n-1}\rho(r')\]
		Now it is possible to simply integrate the field equation starting with:
		\begin{align}
		  \Delta\Phi&=\frac{1}{r^{n-1}}\frac{\partial}{\partial r}\left(r^{n-1}\frac{\partial\Phi}{\partial r}\right)\\
		  &=C(n)G\rho(r)+\lambda\\
		  r^{n-1}\frac{\partial\Phi}{\partial r}&=\int_0^r\textrm{d}r'\left(C(n)G\left(r'\right)^{n-1}\rho\left(r'\right)+\left(r'\right)^{n-1}\lambda\right)\\
		  &=GM+\frac{\lambda}{n}r^n\\
		  \frac{\partial\Phi}{\partial r}&=\frac{GM}{r^{n-1}}+\frac{\lambda r}{n}\\
		  \Phi&=-\frac{GM}{\left(n-2\right)r^{n-2}}+\frac{\lambda r^2}{2n}
		\end{align}
		(b) power-law solutions\\
		Following the calculation one may see that each source term corresponds to an individual power-law:
		\begin{align*}
		  C(n)G\rho(r) ~~~ &\Rightarrow ~~~ -\frac{GM}{\left(n-2\right)r^{n-2}}\\
		  \lambda ~~~ &\Rightarrow ~~~ \frac{\lambda r^2}{2n}
		\end{align*}
		Whereas the \(\lambda\)-term corresponds to a repulsive potential.\\
		(c) equilibrium\\
		To find an equilibrium distance one must set \(\Phi\left(r_\textrm{eq}\right)=0\)
		\end{enumerate}
		\begin{align}
		  \frac{GM}{\left(n-2\right)r_\textrm{eq}^{n-2}}&=\frac{\lambda r_\textrm{eq}^2}{2n}\\
		  \frac{\lambda r_\textrm{eq}^n}{2n}&=\frac{GM}{n-2}
		\end{align}
		from which follows immediatly:
		\begin{equation}
		  r_\textrm{eq}=\sqrt[n]{\frac{GM}{\lambda}\frac{2n}{n-2}}
		\end{equation}
		
        \begin{enumerate}
        \setcounter{enumi}{3}
		\item \question{physics close to the horizon}\\
		Why is it necessary to observe supernovae at the Hubble distance $c/H_0$ to see the dimming in accelerated cosmologies? Please start at considering the curvature scale of the Universe: A convenient quantisation of curvature might be the Ricci-scalar $R = 6H^2(1-q)$ for flat FLRW-models.
		\begin{enumerate}[(a)]
			\item
			\item 
		\end{enumerate}
		
		
		\item \question{measure cosmic acceleration}\\
		The luminosity distance $d_\mathrm{lum}(z)$ in a spatially flat FLRW-universe is given by
		\begin{equation}
		d_\mathrm{lum}(z) = (1+z)\int_0^z\mathrm{d}z^\prime\:\frac{1}{H(z^\prime)}
		\end{equation}
		with the Hubble function $H(z)$. Let's assume that the Universe is filled with a cosmological fluid up to the critical density with a fluid with equation of state $w$, such that the Hubble function is
		\begin{equation}
		H(z) = H_0 (1+z)^\frac{3(1+w)}{2}.
		\end{equation}
		\begin{enumerate}
			\item
					By definition:
					\begin{align*}
	H=\frac{\dot{a}}{a} \text{ and } q=-\frac{\ddot{a}a}{\dot{a}^2}
		\end{align*}
		It follows
		\begin{align*}
		\dot{H}&=\frac{\ddot{a}a-\dot{a}^2}{a^2}=\frac{\ddot{a}a}{a^2}-H^2		
		\end{align*}
		So we get
		\begin{align*}
		\frac{\dot{H}}{H^2}&=\frac{\ddot{a}a}{\dot{a}^2}-1=-q-1\\
		q&=-(\frac{\dot{H}}{H^2}+1)	
		\end{align*}
	
		
		We also have
		\begin{align*}
		H=H_0\cdot(1+z)^{\frac{3(1+w)}{2}}=H_0\cdot a^{\frac{-3(1+w)}{2}}
		\end{align*}
		and
		\begin{align*}
		\dot{H}&=H_0\left(\frac{-3(1+w)}{2}\right)\cdot a^{\frac{-3(1+w)}{2}}\cdot \dot{a}\\
			  &=H_0\cdot a^{\frac{-3(1+w)}{2}}\cdot\frac{\dot{a}}{a}\cdot\left(\frac{-3(1+w)}{2}\right)\\
			  &=H^2\cdot \left(\frac{-3(1+w)}{2}\right)
		\end{align*}	
		so
		\begin{align*}
		q=-\left(\frac{-3(1+w)}{2}+1\right)=\frac{1}{2}(3w+1)
		\end{align*}
		and obviously
		\begin{align*}
		q<0 \text{ for } w<-\frac{1}{3}\\
		q>0 \text{ for } w>-\frac{1}{3}
		\end{align*}	
										
			\item
				First, we consider the case $w=-\frac{1}{3}$ (non-accelerating universe):
				\begin{align*}
			H=H_0(1+z)^{\frac{3(1+w)}{2}}=H_0(1+z)\\
\end{align*}	
\begin{align*}
d_{lum,1}&=(1+z)\int_0^z \! \frac{1}{H(z')} \, \mathrm{d}z'\\
			&=(1+z)\int_0^z \! \frac{1}{H_0(1+z')} \, \mathrm{d}z'\\
			&=\frac{1+z}{H_0}ln(1+z)
\end{align*}						
	Now, we consider the case $w<-\frac{1}{3}$ (accelerating universe):
	\begin{align*}
				d_{lum,2}&=(1+z)\int_0^z \! \frac{1}{H(z')} \, \mathrm{d}z'\\
				&=\frac{1+z}{H_0}\int_0^z \! (1+z')^{\frac{-3(1+w)}{2}} \, \mathrm{d}z'\\
				&=\frac{1+z}{H_0}\left[(1+z')^{\frac{-3(1+w)+2}{2}}\cdot \frac{2}{-3(1+w)+2}\right]_0^z\\
				&=\frac{1+z}{H_0}\left(\frac{2}{-3(1+w)+2}\right)\left[(1+z')^{\frac{-3(1+w)+2}{2}}-1\right]
\end{align*}	
It follows:
$d_{lum_2}(z)>d_{lum_1}(z)$, because the exponent $\frac{-3(1+w)+2}{2}$ is positive $(w<-\frac{1}{3})$,
so $d_{lum_2}(z)$ is growing faster, than the logarithmic function $d_{lum_1}(z)$.		
			\item
			\item
			\item
		\end{enumerate}
		
	\end{enumerate}
\end{document}
